% !TEX TS-program = xelatex
% !TEX encoding = UTF-8

\documentclass[a4paper, 11pt, twocolumn, draft]{article} % use larger type; default would be 10pt

\usepackage{default}

\defaultfontfeatures{Mapping=tex-text}
\setmainfont[Ligatures=TeX]{Linux Libertine}

% \usepackage[parfill]{parskip} % Activate to begin paragraphs with an empty line rather than an indent

\usepackage{graphicx} % support the \includegraphics command and options
\usepackage{bm}

\title{Reinforcement learning with TD(λ) and a Neural Network}
\author{Student 120133998, University of Sheffield}
\date{\today}

\begin{document}

\twocolumn[{
  \begin{@twocolumnfalse}
    \maketitle
    \begin{abstract}

    This report outlines our investigations into the reinforcement learning
    problem by using an example `homing' problem.  We apply the TD algorithm and
    use an artificial neural network to learn Q-values, and then implement
    various extensions to the TD algorithm.  We compare these extensions success
    at learning the homing problem and tune their parameters.

    \end{abstract}
  \end{@twocolumnfalse}
}]

\section{}

\begin{thebibliography}{9}

\end{thebibliography}

\onecolumn \appendix

\section{Tables}
\begin{table}
  \centering
  \begin{tabular}{r | c c c c c c c c c c}
    \hline
      & 0 & 1 & 2 & 3 & 4 & 5 & 6 & 7 & 8 & 9 \\
    \hline
    0 & → & ↓ & ↓ & ↓ & → & ↓ & ↓ & ← & ← & ↓ \\
    1 & ↓ & ↓ & ↓ & ↓ & ↓ & ↓ & ↓ & ← & ↓ & ↓ \\
    2 & ↓ & ↓ & ↓ & ↓ & ↓ & ↓ & ↓ & ↓ & ← & ↓ \\
    3 & ↓ & → & ↓ & ← & ↓ & → & ↓ & ← & ↓ & ↓ \\
    4 & ↓ & ↓ & → & ↓ & ↓ & ↓ & ↓ & ↓ & ← & ← \\
    5 & → & → & → & → & → & \colorbox{pink}{↓} & ← & ← & ↑ & ← \\
    6 & ↑ & → & → & ↑ & ↑ & ↑ & ↑ & ↑ & ← & ← \\
    7 & → & → & ↑ & ← & ↑ & → & ↑ & ← & ← & ← \\
    8 & → & ↑ & ↑ & → & → & → & ↑ & ↑ & ← & ↑ \\
    9 & ↑ & ↑ & ↑ & → & ↑ & ↑ & ↑ & ← & ↑ & ↑ \\
    \hline
  \end{tabular}
  \caption{Directions}
  \label{tab:computed_directions}
\end{table}

\section{Code snippets} \label{sec:code}

\end{document}
% Disable intented paragraphs and add space between them instead.
% \setlength{\parskip}{10pt}
% \setlength{\parindent}{0pt
